\documentclass[a4paper]{article}

\usepackage[T1]{fontenc}
\usepackage{textcomp}
\usepackage[italian]{babel}
\usepackage{hyperref}
\usepackage{amsmath, amssymb, amsthm}
\usepackage{mathtools}
% for \lightning
\usepackage{stmaryrd}
\usepackage{geometry}
\usepackage{tikz-cd}
\usepackage{enumitem}
\usepackage{cancel}

% Remove indentation globally
\setlength{\parindent}{0pt}
% Have blank lines between paragraphs
\usepackage[parfill]{parskip}

\hypersetup{
	colorlinks = true, % links instead of boxes
	urlcolor   = blue, % external hyperlinks
	linkcolor  = blue, % internal links
	citecolor  = red   % citations
}

\newcommand{\R}{\mathbb{R}}
\newcommand{\C}{\mathbb{C}}
\newcommand{\Q}{\mathbb{Q}}
\newcommand{\K}{\mathbb{K}}
\newcommand{\N}{\mathbb{N}}
\newcommand{\A}{\mathbb{A}}
\newcommand{\Z}{\mathbb{Z}}
\renewcommand{\P}{\mathbb{P}}

\newcommand{\id}{\operatorname{id}}
\newcommand{\Hom}{\operatorname{Hom}}

\newcommand{\ssfrac}[2]{
    \raisebox{+0.3ex}{$#1$}
    /
    \raisebox{-0.3ex}{$#2$}
}

\newcommand{\sfrac}[2]{
    \raisebox{+0.3ex}{\scalebox{0.9}{$#1$}}
    /
    \raisebox{-0.3ex}{\scalebox{0.9}{$#2$}}
}

\newcommand\restr[2]{{% we make the whole thing an ordinary symbol
  \left.\kern-\nulldelimiterspace % automatically resize the bar with \right
  #1 % the function
  \vphantom{\big|} % pretend it's a little taller at normal size
  \right|_{#2} % this is the delimiter
  }}

\renewcommand{\labelitemii}{$\circ$}
\renewcommand{\Im}{\operatorname{Im}}

\newcommand\numberthis{\addtocounter{equation}{1}\tag{\theequation}}

\newtheorem{theorem}{Theorem}[section]
\newtheorem{lemma}{Lemma}[section]

\theoremstyle{definition}
\newtheorem{definition}{Definition}[section]

\theoremstyle{definition}
\newtheorem{example}{Example}[section]

\theoremstyle{remark}
\newtheorem*{remark}{Osservazione}

\theoremstyle{definition}
\newtheorem*{exercise}{Esercizio}

\title{Istituzioni di Geometria 2023/2024}
\author{Francesco Minnocci}
\begin{document}
\maketitle
\section*{Quarta Consegna}
\textbf{Esercizio 10.7}
Sia $G$ un gruppo di Lie. Mostra che esiste sempre una metrica riemanniana su $G$ invariante a sinistra, cioè tale che $L_g : G \to G$ sia un'isometria per ogni $g \in G$.
\begin{proof}
Sia $n$ la dimensione di $G$. Possiamo identificare $\mathfrak{g}=T_eG$ con $\R^n$ fissandone una base, ed usare la metrica euclidea standard su $\R^n$ per
definire un prodotto scalare $\langle \cdot, \cdot \rangle_e$ su $T_eG$.

Se poi $g \in G$ e $v, w \in T_gG$, possiamo estendere il prodotto scalare definito su $T_eG$ per traslazione, cioè ponendo
\[
\langle v, w \rangle_g = \langle (dL_{g^{-1}})_g(v), (dL_{g^{-1}})_g(w) \rangle_e
\]
Per costruzione, $\langle \cdot, \cdot \rangle_g$ è invariante a sinistra:
\[
\langle (dL_g)_h(v), (dL_g)_h(w) \rangle_{gh} = \langle v, w \rangle_h
\]
per ogni $g, h \in G$ e $v, w \in T_hG$.

\end{proof}

\textbf{Esercizio 10.2}
%Considera il piano iperbolico nel modello del semipiano:
%H2 =
%�
%(x, y) ∈R2 | y > 0
%�
%,
%g = 1
%y 2 gE.
%Calcola l’area del dominio
%[−a, a] × [b, ∞)
%per ogni a, b > 0. L’area è ovviamente quella indotta dalla forma volume della
%varietà riemanniana H2.

Considera il piano iperbolico nel modello del semipiano:
\[
H^2 = \{(x, y) \in \R^2 \mid y > 0\}, \quad g = \frac{1}{y^2} \, g_E
\]
Calcola l'area del dominio
\[
[-a, a] \times [b, \infty)
\] per ogni $a, b > 0$. L'area è ovviamente quella indotta dalla forma volume della varietà riemanniana $H^2$.
\begin{proof}
La forma volume indotta dalla metrica $g$ sul piano iperbolico è
\[
    \omega = \frac{1}{y^2} \, dx \wedge dy
.\]
L'area di $A:=[-a, a] \times [b, \infty)$ è quindi
\[
    \int_A \omega = \int_A \frac{1}{y^2} \, dx \wedge dy
    \int_{b}^{\infty} \int_{-a}^{a} \frac{1}{y^2} \, dx \, dy = 2a\cdot \int_{b}^{\infty} \frac{1}{y^2} \, dy = 2a \left[ -\frac{1}{y} \right]_{b}^{\infty} = \frac{2a}{b}
.\]
\end{proof}
\end{document}
