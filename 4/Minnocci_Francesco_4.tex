\documentclass[a4paper]{article}

\usepackage[T1]{fontenc}
\usepackage{textcomp}
\usepackage[italian]{babel}
\usepackage{hyperref}
\usepackage{amsmath, amssymb, amsthm}
\usepackage{mathtools}
% for \lightning
\usepackage{stmaryrd}
\usepackage{geometry}
\usepackage{tikz-cd}
\usepackage{enumitem}
\usepackage{cancel}

% Remove indentation globally
\setlength{\parindent}{0pt}
% Have blank lines between paragraphs
\usepackage[parfill]{parskip}

\hypersetup{
	colorlinks = true, % links instead of boxes
	urlcolor   = blue, % external hyperlinks
	linkcolor  = blue, % internal links
	citecolor  = red   % citations
}

\newcommand{\R}{\mathbb{R}}
\newcommand{\C}{\mathbb{C}}
\newcommand{\Q}{\mathbb{Q}}
\newcommand{\K}{\mathbb{K}}
\newcommand{\N}{\mathbb{N}}
\newcommand{\A}{\mathbb{A}}
\newcommand{\Z}{\mathbb{Z}}
\renewcommand{\P}{\mathbb{P}}

\newcommand{\id}{\operatorname{id}}
\newcommand{\Hom}{\operatorname{Hom}}

\newcommand{\ssfrac}[2]{
    \raisebox{+0.3ex}{$#1$}
    /
    \raisebox{-0.3ex}{$#2$}
}

\newcommand{\sfrac}[2]{
    \raisebox{+0.3ex}{\scalebox{0.9}{$#1$}}
    /
    \raisebox{-0.3ex}{\scalebox{0.9}{$#2$}}
}

\newcommand\restr[2]{{% we make the whole thing an ordinary symbol
  \left.\kern-\nulldelimiterspace % automatically resize the bar with \right
  #1 % the function
  \vphantom{\big|} % pretend it's a little taller at normal size
  \right|_{#2} % this is the delimiter
  }}

\renewcommand{\labelitemii}{$\circ$}
\renewcommand{\Im}{\operatorname{Im}}

\newcommand\numberthis{\addtocounter{equation}{1}\tag{\theequation}}

\newtheorem{theorem}{Theorem}[section]
\newtheorem{lemma}{Lemma}[section]

\theoremstyle{definition}
\newtheorem{definition}{Definition}[section]

\theoremstyle{definition}
\newtheorem{example}{Example}[section]

\theoremstyle{remark}
\newtheorem*{remark}{Osservazione}

\theoremstyle{definition}
\newtheorem*{exercise}{Esercizio}

\title{Istituzioni di Geometria 2023/2024}
\author{Francesco Minnocci}
\begin{document}
\maketitle
\section*{Quarta Consegna}
\textbf{Esercizio 10.1}
Considera lo spazio iperbolico nel modello del semispazio:
\[
    H^n = \{x \in \R^n \mid x_n > 0\}, \quad g(x) = \frac{1}{x_n^2} \, g_E(x)
\]
dove $g_E$ è il tensore euclideo. Mostra che le mappe seguenti sono isometrie per la varietà riemanniana $H^n$:
\begin{itemize}
    \item $f(x) = x + b$, con $b = (b_1, \ldots, b_{n-1}, 0)$;
    \item $f(x) = \lambda x$ con $\lambda > 0$.
\end{itemize}
Deduci che il gruppo di isometrie $\operatorname{Isom}(H^n)$ di $H^n$ agisce transitivamente sulla varietà riemanniana $H^n$.

\begin{proof}
    Sia $x \in H^n$ e $v, w \in T_xH^n$. Allora
    \[
        \langle v, w \rangle_x = \frac{1}{x_n^2} \, \langle v, w \rangle_E
    \]
    dove $\langle \cdot, \cdot \rangle_E$ è il prodotto scalare euclideo. Consideriamo la mappa $f(x) = x + b$; poichè $df_x(v) = v$ per ogni $v \in T_xH^n$, abbiamo
    \[
        \langle df_x(v), df_x(w) \rangle_{f(x)} = \langle v, w \rangle_{x + b} = \frac{1}{x_n^2} \, \langle v, w \rangle_E
    \]
    per ogni $v, w \in T_xH^n$. Dunque $f$ è un'isometria.

    Presa invece la mappa $h(x) = \lambda x$ con $\lambda > 0$, si ha $dh_x= \lambda \id_{T_xH^n}$, dunque
    \[
        \langle dh_x(v), dh_x(w) \rangle_{h(x)} = \langle \lambda v, \lambda w \rangle_{\lambda x} = \frac{1}{(\lambda x_n)^2} \, \langle \lambda v, \lambda w \rangle_E = \frac{1}{x_n^2} \, \langle v, w \rangle_E
    \]
    per ogni $v, w \in T_xH^n$ per bilinearità del prodotto scalare. Quindi anche $f$ è un'isometria.

    Infine, presi $x, y \in H^n$, esiste $\lambda > 0$ tale che $y_n = \lambda x_n$ (ovvero $\lambda = \frac{y_n}{x_n}$).
    Allora, posto \[b=(y_1 - \lambda x_1, \ldots, y_{n-1} - \lambda x_{n-1}, 0),\] l'isometria \[y = \lambda x + b\] manda $x$ in $y$, per cui il gruppo $\operatorname{Isom}(H^n)$ agisce transitivamente su $H^n$.
\end{proof}

\textbf{Esercizio 10.2}
Considera il piano iperbolico nel modello del semipiano:
\[
H^2 = \{(x, y) \in \R^2 \mid y > 0\}, \quad g = \frac{1}{y^2} \, g_E
\]
Calcola l'area del dominio
\[
[-a, a] \times [b, \infty)
\] per ogni $a, b > 0$. L'area è ovviamente quella indotta dalla forma volume della varietà riemanniana $H^2$.
\begin{proof}
La forma volume indotta dalla metrica $g$ sul piano iperbolico è
\[
    \omega = \frac{1}{y^2} \, dx \wedge dy
.\]
L'area di $A:=[-a, a] \times [b, \infty)$ è quindi
\[
    \int_A \omega = \int_A \frac{1}{y^2} \, dx \wedge dy
    \int_{b}^{\infty} \int_{-a}^{a} \frac{1}{y^2} \, dx \, dy = 2a\cdot \int_{b}^{\infty} \frac{1}{y^2} \, dy = 2a \left[ -\frac{1}{y} \right]_{b}^{\infty} = \frac{2a}{b}
.\]
\end{proof}
\textbf{Esercizio 10.7}
Sia $G$ un gruppo di Lie. Mostra che esiste sempre una metrica riemanniana su $G$ invariante a sinistra, cioè tale che $L_g : G \to G$ sia un'isometria per ogni $g \in G$.
\begin{proof}
Sia $n$ la dimensione di $G$. Possiamo identificare $\mathfrak{g}=T_eG$ con $\R^n$ fissandone una base, ed usare la metrica euclidea standard su $\R^n$ per
definire un prodotto scalare $\langle \cdot, \cdot \rangle_e$ su $T_eG$.

Se poi $g \in G$ e $v, w \in T_gG$, possiamo estendere il prodotto scalare definito su $T_eG$ per traslazione, cioè ponendo
\[
\langle v, w \rangle_g = \langle (dL_{g^{-1}})_g(v), (dL_{g^{-1}})_g(w) \rangle_e
\]
Per costruzione, $\langle \cdot, \cdot \rangle_g$ è invariante a sinistra:
\[
\langle (dL_g)_h(v), (dL_g)_h(w) \rangle_{gh} = \langle v, w \rangle_h
\]
per ogni $g, h \in G$ e $v, w \in T_hG$.

\end{proof}

\textbf{Esercizio 11.4}
Consideriamo la connessione $\nabla$ su $\R^3$ con simboli di Christoffel
\begin{align*}
    \Gamma^3_{12} = \Gamma^1_{23} = \Gamma^2_{31} = 1, \\
    \Gamma^3_{21} = \Gamma^1_{32} = \Gamma^2_{13} = -1
\end{align*}
e tutti gli altri simboli di Christoffel nulli. Mostra che questa connessione è compatibile con il tensore metrico euclideo $g$, ma non è simmetrica. Quali sono le geodetiche?
\begin{proof}
Per la Proposizione 9.3.5 delle dispense, $\nabla$ è compatibile con $g$ se e solo se
\[
    \frac{\partial g_{ij}}{\partial x^k} = \Gamma^l_{ki} g_{lj} + \Gamma^l_{kj} g_{li}
.\]
Poiché $g$ è il tensore metrico euclideo, $g_{ij} = \delta_{ij}$, e quindi la condizione di compatibilità diventa
\[
    0 = \Gamma^j_{ki}g_{jj} + \Gamma^i_{kj}g_{ii}=\Gamma^j_{ki}+\Gamma^i_{kj}
.\]
D'altronde, se $i,j,k$ non sono tutti distinti allora $\Gamma^j_{ki}=\Gamma^i_{kj}=0$ per ipotesi, mentre se lo sono allora
$\Gamma^j_{ki}=-\Gamma^i_{kj}$. Dunque $\nabla$ è compatibile con $g$.

Inoltre $\nabla$ non è simmetrica perché $\Gamma^3_{12} \neq \Gamma^3_{21}$.

Infine, se $x(t)$ è la geodetica massimale passante per $x_0$ in direzione $v$, allora $x(t)$ risolve
\[
\begin{cases}
x(0) = x_0, \\
\dot{x}(t) = v, \\
\frac{\partial^2x^k}{\partial t^2} + \frac{\partial x^i}{\partial t}\frac{\partial x^j}{\partial t}\Gamma^k_{ij} = 0
\end{cases}
\]
per $k=1,2,3$. Visto che $\Gamma^1_{23} = -\Gamma^1_{32}$, $\Gamma^2_{31} = -\Gamma^2_{13}$ e $\Gamma^3_{12} = -\Gamma^3_{21}$, questo implica \[\frac{\partial^2 x^1}{\partial t^2}
= \frac{\partial^2 x^2}{\partial t^2} = \frac{\partial^2 x^3}{\partial t^2} = 0,\] e quindi $x(t)=x_0+t v$. In conclusione, le geodetiche sono tutte e solo le rette.
\end{proof}

\textbf{Esercizio 11.6}
Consideriamo il modello dell'iperboloide $I^n \subset \R^{n,1}$ dello spazio iperbolico. Mostra che per ogni $p, q \in I^n$ vale l'uguaglianza
\[
    \cosh d(p, q) = -\langle p, q \rangle
.\]

\end{document}
