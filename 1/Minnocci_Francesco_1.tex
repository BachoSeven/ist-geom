\documentclass[a4paper]{article}

\usepackage[T1]{fontenc}
\usepackage{textcomp}
\usepackage[italian]{babel}
\usepackage{hyperref}
\usepackage{amsmath, amssymb, amsthm}
\usepackage{mathtools}
% for \lightning
\usepackage{stmaryrd}
\usepackage{geometry}
\usepackage{tikz-cd}
\usepackage{enumitem}

% Remove indentation globally
\setlength{\parindent}{0pt}
% Have blank lines between paragraphs
\usepackage[parfill]{parskip}

\hypersetup{
	colorlinks = true, % links instead of boxes
	urlcolor   = blue, % external hyperlinks
	linkcolor  = blue, % internal links
	citecolor  = red   % citations
}

\newcommand{\R}{\mathbb{R}}
\newcommand{\C}{\mathbb{C}}
\newcommand{\Q}{\mathbb{Q}}
\newcommand{\K}{\mathbb{K}}
\newcommand{\N}{\mathbb{N}}
\newcommand{\A}{\mathbb{A}}
\newcommand{\Z}{\mathbb{Z}}
\renewcommand{\P}{\mathbb{P}}

\newcommand{\id}{\operatorname{id}}
\newcommand{\Hom}{\operatorname{Hom}}

\newcommand{\ssfrac}[2]{
    \raisebox{+0.3ex}{$#1$}
    /
    \raisebox{-0.3ex}{$#2$}
}

\newcommand{\sfrac}[2]{
    \raisebox{+0.3ex}{\scalebox{0.9}{$#1$}}
    /
    \raisebox{-0.3ex}{\scalebox{0.9}{$#2$}}
}

\newcommand\restr[2]{{% we make the whole thing an ordinary symbol
  \left.\kern-\nulldelimiterspace % automatically resize the bar with \right
  #1 % the function
  \vphantom{\big|} % pretend it's a little taller at normal size
  \right|_{#2} % this is the delimiter
  }}

\renewcommand{\labelitemii}{$\circ$}
\renewcommand{\Im}{\operatorname{Im}}

\newcommand\numberthis{\addtocounter{equation}{1}\tag{\theequation}}

\newtheorem{theorem}{Theorem}[section]
\newtheorem{lemma}{Lemma}[section]

\theoremstyle{definition}
\newtheorem{definition}{Definition}[section]

\theoremstyle{definition}
\newtheorem{example}{Example}[section]

\theoremstyle{remark}
\newtheorem*{remark}{Osservazione}

\theoremstyle{definition}
\newtheorem*{exercise}{Esercizio}

\title{Istituzioni di Geometria 2023/2024}
\author{Francesco Minnocci}
\begin{document}
\maketitle
\section*{Prima Consegna}
\textbf{Esercizio 1.1}
Costruisci due atlanti lisci non compatibili su $\R$. Mostra che le due varietá lisce che ne risultano sono però diffeomorfe.
\begin{proof}
    Consideriamo due atlanti su $\R$ formati da una sola carta: quello standard dato dall'identità ed $\mathcal{A} = \{ f:\: \R\to\R \}$, con
    \[
        f(x) = \begin{cases}
            x   & \text{se } x \leq 0, \\
            x^2 & \text{se } x > 0.
        \end{cases}
    \]
    Quest'ultimo è un'atlante liscio in quanto l'unica mappa di transizione $f \circ f ^{-1} =  \id_{\R}$ è liscia, tuttavia non è compatibile con l'atlante standard: se
    lo fosse, allora la mappa di transizione
    \[
        f \circ \id_{\R}^{-1} = f
    \]
    dovrebbe essere liscia, ma non lo è in $x = 0$.

    Mostriamo infine che $f$ è un diffeomorfismo tra $(\R, \mathcal{A})$ ed $\R$ con l'atlante standard: $f$ è una funzione liscia tra varietà perché in carte diventa
    \[
        \id_{\R} \circ f \circ f^{-1} = \id_{\R}
        .\]
    Analogamente, la sua inversa
    \[
        f^{-1}(x) = \begin{cases}
            x        & \text{se } x \leq 0, \\
            \sqrt{x} & \text{se } x > 0
        \end{cases}
    \]
    è liscia visto che \[
        f \circ f^{-1} \circ \id_{\R}^{-1} = \id_{\R}.
    \]
    \begin{remark}
        La costruzione funziona anche sostituendo $f$ con un qualsiasi omeomorfismo di $\R$ in sè che non sia $C^\infty$.
    \end{remark}
\end{proof}
\textbf{Esercizio 1.2} (svolto in collaborazione con Eva Silvestri)
Mostra che la mappa
\[
    f:\:S^n \to \R\P^n, \quad (x_0, \ldots, x_n) \mapsto [x_0 : \ldots : x_n]
\]
è liscia.

\begin{proof}
    Sia $(x_0, \ldots, x_n) \in S^n$, e fissiamo un indice $i$ tale che $x_i \neq 0$. Senza perdità di generalità, supponiamo $x_i>0$ (le coordinate non cambiano nel caso $x_i < 0$). Allora, $x_i$ è contenuto nell'aperto
    \[
        U_i^+ := \{ (x_0 , \ldots , x_n) \in S^n \mid x_i > 0 \},
    \]
    che identifichiamo con la palla aperta unitaria $B^n\subset\R^n$ tramite la carta
    \begin{align*}
        \varphi_i^+ : U_i^+        & \to B^n                                    \\
        \quad (x_0 , \ldots , x_n) & \mapsto (x_0,\ldots,\hat{x}_i,\ldots,x_n).
    \end{align*}
    Inoltre, notiamo che $f$ manda $U_i^+$ nella carta affine
    \[
        V_i := \{ [x_0 : \ldots : x_n] \in \R\P^n \mid x_i \neq 0 \}
        ,\]
    che è omeomorfa ad $\R^n$ attraverso la mappa
    \begin{align*}
        \psi_i : V_i               & \longrightarrow \R^n                                                                                                   \\
        \quad [x_0 : \ldots : x_n] & \longmapsto \left( \frac{x_0}{x_i}, \ldots, \frac{x_{i-1}}{x_i}, \frac{x_{i+1}}{x_i}, \ldots, \frac{x_n}{x_i} \right).
    \end{align*}
    Mostriamo ora che
    \[
        F \coloneqq  \psi_i \circ f \circ (\varphi_i^+)^{-1} : B^n \to \R^n
    \]
    è ben definita e liscia: l'inversa di $\varphi_i^+$ è data da
    \begin{align*}
        (\varphi_i^+)^{-1} : B^n & \to U_i^+                                                                                           \\
        y                        & \mapsto \left( y_1, \ldots, y_{i-1}, \sqrt{1 - {\lVert y \rVert}^2 }, y_{i+1}, \ldots, y_n \right).
    \end{align*}
    Inoltre, dato che $\lVert y \rVert < 1$ si ha che
    \[
        F(y) = \left( \frac{y_1}{\sqrt{1 - {\lVert y \rVert}^2 }}, \ldots, \frac{y_n}{\sqrt{1 - {\lVert y \rVert}^2 }} \right)
    \]
    è un'onesta composizione di mappe lisce, ed è quindi liscia.
\end{proof}

\textbf{Esercizio 1.3} (svolto in collaborazione con Eva Silvestri)
Costruisci un diffeomorfismo tra $S^1$ ed $\R\P^1$.

\begin{proof}
    Siano $N=(0,1)$ ed $S=(0,-1)$ i poli sud e nord di $S^1$, e definiamo le mappe $f_i$ per $i=1,2$ come
    \begin{align*}
        f_1 & =\phi_N^{-1}\circ D_1 :\: U_1\subset\R\P^1\to S^1\setminus N, \\
        f_2 & =\phi_S^{-1}\circ D_2 :\: U_2\subset\R\P^1\to S^1\setminus S.
    \end{align*}
    dove $\phi_S$ e $\phi_N$ sono le proiezioni stereografiche, cioè
    \begin{align*}
        \phi_N (x,y) & = \frac{2x}{1-y}, \\
        \phi_S (x,y) & = \frac{2x}{1+y},
    \end{align*}
    e $D_i:\:U_i\to\R$ per $i=1,2$ sono le carte di $\R\P^1$
    \begin{align*}
        D_1([x : y]) & = \frac{y}{x}, \\
        D_2([x : y]) & = \frac{x}{y}.
    \end{align*}
    Scriviamo esplicitamente le parametrizzazioni di $S^1$:
    \begin{align*}
        \phi_N^{-1}(t) & = \left( \frac{t}{1+t^2}, \frac{t^2-1}{1+t^2} \right), \\
        \phi_S^{-1}(t) & = \left( \frac{t}{1+t^2}, \frac{1-t^2}{1+t^2} \right).
    \end{align*}
    Visto che le $f_i$ coincidono sull'intersezione dei domini $S^1 \setminus \{ N, S \}$, abbiamo costruito per casi un diffeomorfismo $f$ tra $\R\P^1$ ed $S^1$, posto che le $f_i$ siano lisce: l' inversa
    $g :\; S^1 \to \R\P^1$ è analogamente definita per casi come
    \[
        g(x,y) = \begin{cases}
            (H_1\circ\phi_N)(x,y)=[1-y:2x] & \text{se } y \neq 1,  \\
            (H_2\circ\phi_S)(x,y)=[2x:1+y] & \text{se } y \neq -1,
        \end{cases}
    \]
    dove $H_i$ è l'inversa di $D_i$ per $i=1,2$.

    Infine, usando come atlanti per $S^1$ e $\R\P^1$ quelli descritti sopra, $f$ e $g$ sono chiaramente lisce perché inducono l'identità su $\R$ in carte.
\end{proof}
\textbf{Esercizio 2.1}
Sia $p(z)\in\C[z]$ polinomio di grado $d\geq1$. Considera l'insieme
$S = \{z \mid p^{\prime} (z) = 0\}.$ Mostra che la mappa
\begin{align*}
    p: \C \setminus p^{-1}(p(S)) & \to \C \setminus p(S), \\
    z                            & \mapsto p(z)
\end{align*}
è un rivestimento liscio di grado d.
\begin{proof}
    Innanzitutto, per il criterio della derivata le fibre della mappa in questione sono tutte di cardinalità $d$. Inoltre, la mappa è un diffeomorfismo locale per il teorema d'invertibilità locale, sempre per la scelta del dominio effettuata. Ora, preso $y\in p(S)$, sia $p^{-1}(x)=\{x_1, \dots , x_d\}$, e siano $V_i$ degli intorni aperti degli $x_i$ tali che $p$ è un diffeomorfismo su $V_i$. A meno di restringerli, possiamo supporre che i $V_i$ siano disgiunti, e posto $U\coloneqq \bigcap_{i=1}^d p(V_i)$, si ha che
    \[
        p^{-1}(U) = \bigsqcup_{i=1}^d V_i.
    \]
    Abbiamo quindi mostrato che $U$ è un intorno ben rivestito di $y$ tale che $p: V_i \to U$ è un diffeomorfismo per $i = 1, \ldots, d$, e quindi $p$ è un rivestimento liscio di grado $d$.
\end{proof}
\textbf{Esercizio 2.2} (svolto in collaborazione con Lorenzo Femia)

Considera il gruppo $\Gamma < \operatorname{Isom}(\R^3)$ generato da:
\begin{align*}
    f(x,y,z) = ( & x+1,y,z), \qquad g(x,y,z) = (x,y+1,z), \\
                 & h(x,y,z) = (-x,-y,z+1).
\end{align*}
Mostra che l'azione è libera e propriamente discontinua, e che la varietà $\R³/\Gamma$ è compatta ed orientabile ma non omeomorfa al 3-toro. Mostra che questa varietà ha un
rivestimento doppio diffeomorfo al 3-toro.
\begin{proof}
    Osserviamo che valgono le seguenti relazioni:
    \[
        \begin{cases}
            f \circ g = g\circ f,      \\
            h \circ f = f^{-1}\circ h, \\
            h \circ g = g^{-1}\circ h.
        \end{cases}
    \]
    Da queste segue che ogni elemento $\gamma$ di $\Gamma$ si scrive in modo unico come
    \[
        \gamma = f^{k} \circ g^{l} \circ h^{m}
    \]
    per qualche $k,l,m\in\Z$. Vediamo che lo stabilizzatore di qualsiasi punto $(x,y,z)\in\R^3$ è banale:
    \begin{equation}\label{eq:orbit}
        \gamma \cdot (x,y,z) = f^{k} \circ g^{l} \circ h^{m} \cdot (x,y,z) = ((-1)^m x+k,(-1)^m y+l,z+m),
    \end{equation}
    e quindi $\gamma \cdot (x,y,z) = (x,y,z)$ se e solo se $k=l=m=0$.

    Per vedere che l'azione è propriamente discontinua, presi due punti $x,y\in\R^3$ scegliamo come loro intorni due palle aperte di raggio $\frac{1}{4}$. Allora, la formula
    \eqref{eq:orbit} mostra che l'orbita di $x$ è discreta, e quindi la sua intersezione con il compatto $\overline{B_{\frac{1}{4}}(y)}$ è finita, il che mostra che l'azione è propriamente discontinua.

    Per quanto riguarda la compattezza, osserviamo che $D\coloneqq [-\frac{1}{2},\frac{1}{2}]\times [-\frac{1}{2},\frac{1}{2}]\times[0,1]$ è un dominio fondamentale per l'azione di $\Gamma$, e quindi
    \[
        \ssfrac{\R^3}{\Gamma}\simeq \ssfrac{D}{\Gamma}
    \]
    è immagine del compatto $D$ mediante la mappa continua di proiezione, che ne mostra la compattezza.

    L'orientabilità di $\sfrac{\R^3}{\Gamma}$ segue dal fatto che $\Gamma$ preserva l'orientazione. Infatti, visto che lavoriamo su $\R^3$ possiamo calcolare i differenziali dei generatori di $\Gamma$ con lo Jacobiano, ed hanno tutti determinante 1:
    \[
        \begin{cases}
            df_p = \begin{pmatrix}
                1 & 0 & 0 \\
                0 & 1 & 0 \\
                0 & 0 & 1
            \end{pmatrix}, \\
            dg_p = \begin{pmatrix}
                1 & 0 & 0 \\
                0 & 1 & 0 \\
                0 & 0 & 1
            \end{pmatrix}, \\
            dh_p = \begin{pmatrix}
                -1 & 0 & 0 \\
                0  & -1 & 0 \\
                0  & 0  & 1
            \end{pmatrix}.
        \end{cases}
    \]

    Essendo il rivestimento $\R^3 \to \sfrac{\R^3}{\Gamma}$ regolare, il gruppo fondamentale di $\sfrac{\R^3}{\Gamma}$ è isomorfo a $\Gamma$, che non è abeliano per le relazioni esibite sopra. D'altra parte, $\pi_1(T^3) = \Z^3$ è abeliano, e quindi la varietà $\sfrac{\R^3}{\Gamma}$ non è omeomorfa al 3-toro.

    Infine, il sottogruppo normale $N=\langle f,g,h^2 \rangle$ di $\Gamma$ ha indice 2, quindi induce un rivestimento di grado 2
    \[
        \ssfrac{\R^3}{N} \to \ssfrac{\R^3}{\Gamma}.
    \]
    L'azione di $N$ su $\R^3$ coincide con quella che dà il 3-toro, tranne che sull'asse $z$ dove trasla di 2 invece che di 1. Per mostrare $N$ è diffeomorfo al 3-toro basta quindi
    osservare che $D$ è anche un dominio fondamentale per l'azione che dà il 3-toro, e che un dominio fondamentale per l'azione di $N$ è $D' = [-\frac{1}{2},\frac{1}{2}]\times [-\frac{1}{2},\frac{1}{2}]\times[0,2]$, diffeomorfo a $D$ tramite
    $(x,y,z)\mapsto (x,y,2z)$. Questo induce il diffeomorfismo cercato.
\end{proof}
\textbf{Esercizio 2.5}
Sia $M$ compatta ed $N$ connessa. Se $\dim M = \dim N$, mostra che ogni embedding $M\to N$ è un diffeomorfismo.
\begin{proof}
    Sia $f:M\to N$ un embedding. Poiché $M$ è compatto, $f(M)$ è compatto in $N$, e quindi chiuso; inoltre $f(M)$ è aperto in $N$ perché $f$ è liscia, ma essendo $N$ connessa, $f(M) = N$.

    Inoltre, $f$ è un'immersione iniettiva tra spazi della stessa dimensione, che per il teorema d'invertibilità locale implica che $f$ è un diffeomorfismo locale.

    Infine, per definizione $f$ è un omeomorfismo con l'immagine, e l'inversa continua $f^{-1}: N \to M$ è liscia in quanto $f$ è un diffeomorfismo locale.
\end{proof}
\textbf{Esercizio 3.1}
Siano $v,v',w,w'\in V^*$ covettori non nulli.
\begin{enumerate}
    \item Se $v$ e $v'$ sono indipendenti, allora $v\otimes w$ e $v'\otimes w'$ sono vettori indipendenti in $\mathcal{T}^2(V)$.
    \item Se inoltre anche $w$ e $w'$ sono indipendenti, allora
        $$
        v\otimes w+v'\otimes w' \in \mathcal{T}^2(V)
        $$
        non è un elemento puro.
\end{enumerate}
\begin{proof}\

    \begin{enumerate}
        \item Supponiamo per assurdo che esistano $\lambda,\mu\in\R$ tali che
              \[
                  \lambda(v\otimes w) + \mu(v'\otimes w') = 0,
              \]
              e supponiamo senza perdita di generalità che $\lambda\neq 0$. Allora, valutando entrambi i membri in $(x,y)\in V\times V$ otteniamo
              \[
                  \lambda v(x) w(y) + \mu v'(x) w'(y) = 0,
              \]
              ma per ipotesi esiste $\overline{y}\in V$ tale che $w(\overline{y})\neq 0$, e quindi
              \[
                  (\lambda w(\overline{y})) v(x) + (\mu w'(\overline{y})) v'(x) = 0
              \]
        contraddice l'indipendenza di $v$ e $v'$ (per la genericità di $x$).
        \item Supponiamo per assurdo che $v\otimes w + v'\otimes w'$ sia un elemento puro, cioè che esistano $v'',w''\in V$ tale che
              \begin{equation}\label{eq:assurdo}
                  v''\otimes w'' = v\otimes w + v'\otimes w' .
              \end{equation}
              Ora, completando $\{v,v'\}$ e $\{w,w'\}$ a delle basi $\mathcal{B}=\{v_i\}$ e $\mathcal{C}=\{w_j\}$ di $V^*$, otteniamo una base di $\mathcal{T}^2(V)$ data da tutti i possibili prodotti tensoriali di
              un elemento di $\mathcal{B}$ con uno di $\mathcal{C}$. In particolare, scrivendo in tale base il tensore $v''\otimes w''$ si ha
              \[
               v''\otimes w'' = \sum_{i,j} \lambda_i\mu_j v_i\otimes w_j,
              \]
              che per l'unicità di tale scrittura implica che $\mu_1\lambda_1 = \mu_2\lambda_2 = 1$ (visto che $v_1 = v$, $v_2 = v'$, $w_1 = w$ e $w_2 = w'$). Ma allora in
              particolare \(
              \mu_1\lambda_2\neq0 ,\) che contraddice \eqref{eq:assurdo}.
    \end{enumerate}
\end{proof}
\textbf{Esercizio 3.2}
Considera l'isomorfismo canonico $\mathcal{T}_1^1(V) = \Hom(V,V)$. Mostra che questo isomorfismo manda gli elementi puri in tutti e soli omomorfismi di rango $\leq 1$.
\begin{proof}
    Sia $v\otimes v^*\in\mathcal{T}_1^1(V)$ un elemento puro non nullo. Allora, detto $\varphi$ l'isomorfismo canonico, si ha
    \[
        \varphi(v\otimes v^*)(x) = v^*(x)v
    \]
    per ogni $x\in V$, e quindi $\varphi(v\otimes v^*)$ è un endomorfismo di rango 1.

    Viceversa, se $\phi\in\Hom(V,V)$ è un endomorfismo di rango 1, allora presa una base $\{v_i\}$ di $V$ questa verrà mandata nello span
    di un certo vettore $v\in V$, ovvero $\phi(v_i) = \lambda_i v$ per certi $\lambda_i\in\R$.
    Quindi, ponendo $v^*(x) = \sum_i \lambda_i v_i^*(x)$ con $\{v_i^*\}$ base duale di $\{v_i\}$, si ha che $\phi$ è l'immagine di $v\otimes v^*$ tramite l'isomorfismo $\varphi$ in
    quanto coincidono sulla base scelta:
    \[
        \varphi(v\otimes v^*)(v_i) = v^*(v_i)v = \lambda_i v = \phi(v_i).
    \]
\end{proof}
\textbf{Esercizio 3.3}
Siano $v^1, \ldots, v^k \in V^*$. Mostra che questi vettori sono indipendenti se e solo se $v^1\wedge \ldots \wedge v^k \neq 0$.
\begin{proof}
    Supponiamo che $v^1, \ldots, v^k$ siano dipendenti. Allora esistono $\lambda_1, \ldots, \lambda_k \in \R$ non tutti nulli tali che
    \[
        v^1 = \lambda_2 v^2 + \ldots + \lambda_k v^k,
    \]
    e sviluppando per multilinearità lungo la prima componente otteniamo che
    \[
        v^1\wedge v^2 \wedge \ldots \wedge v^k =  \sum_{i=2}^k \lambda_i v^i \wedge \ldots \wedge v^i \wedge \ldots \wedge v^k = 0.
    \]
    Viceversa, se $v^1, \ldots, v^k$ sono indipendenti, li possiamo completare a una base $\{v^i\}$ di $V^*$, e detta $\{v_i\}$ la base duale associata, abbiamo
    \[
        (v^1\wedge \ldots \wedge v^k) (v_1, \ldots, v_k) = 1 .
    \]
\end{proof}
\end{document}
