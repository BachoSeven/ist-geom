\documentclass[a4paper]{article}

\usepackage[T1]{fontenc}
\usepackage{textcomp}
\usepackage[italian]{babel}
\usepackage{hyperref}
\usepackage{amsmath, amssymb, amsthm}
\usepackage{mathtools}
% for \lightning
\usepackage{stmaryrd}
\usepackage{geometry}
\usepackage{tikz-cd}
\usepackage{enumitem}
\usepackage{cancel}

% Remove indentation globally
\setlength{\parindent}{0pt}
% Have blank lines between paragraphs
\usepackage[parfill]{parskip}

\hypersetup{
	colorlinks = true, % links instead of boxes
	urlcolor   = blue, % external hyperlinks
	linkcolor  = blue, % internal links
	citecolor  = red   % citations
}

\newcommand{\R}{\mathbb{R}}
\newcommand{\C}{\mathbb{C}}
\newcommand{\Q}{\mathbb{Q}}
\newcommand{\K}{\mathbb{K}}
\newcommand{\N}{\mathbb{N}}
\newcommand{\A}{\mathbb{A}}
\newcommand{\Z}{\mathbb{Z}}
\renewcommand{\P}{\mathbb{P}}

\newcommand{\id}{\operatorname{id}}
\newcommand{\Hom}{\operatorname{Hom}}

\newcommand{\ssfrac}[2]{
    \raisebox{+0.3ex}{$#1$}
    /
    \raisebox{-0.3ex}{$#2$}
}

\newcommand{\sfrac}[2]{
    \raisebox{+0.3ex}{\scalebox{0.9}{$#1$}}
    /
    \raisebox{-0.3ex}{\scalebox{0.9}{$#2$}}
}

\newcommand\restr[2]{{% we make the whole thing an ordinary symbol
  \left.\kern-\nulldelimiterspace % automatically resize the bar with \right
  #1 % the function
  \vphantom{\big|} % pretend it's a little taller at normal size
  \right|_{#2} % this is the delimiter
  }}

\renewcommand{\labelitemii}{$\circ$}
\renewcommand{\Im}{\operatorname{Im}}

\newcommand\numberthis{\addtocounter{equation}{1}\tag{\theequation}}

\newtheorem{theorem}{Theorem}[section]
\newtheorem{lemma}{Lemma}[section]

\theoremstyle{definition}
\newtheorem{definition}{Definition}[section]

\theoremstyle{definition}
\newtheorem{example}{Example}[section]

\theoremstyle{remark}
\newtheorem*{remark}{Osservazione}

\theoremstyle{definition}
\newtheorem*{exercise}{Esercizio}

\title{Istituzioni di Geometria 2023/2024}
\author{Francesco Minnocci}
\begin{document}
\maketitle
\section*{Seconda Consegna}
\textbf{Esercizio 4.2}
Mostra che il fibrato tangente $TM$ di una varietà $M$ è sempre orientabile, anche se $M$ non lo è.

\begin{proof}
    Sia $\mathcal{A}=\{\phi_i:U_i\to V_i\}$ un atlante per $M$. Sfruttando la struttura liscia dei fibrati tangenti, possiamo costruire un atlante per $TM$ tramite i differenziali delle $\phi_i$,
    ovvero $\mathcal{A}^\prime=\{(\phi)_*:TU_i\to TV_i=V_i\times\R^n\}$; infatti le mappe di transizione sono lisce per funtorialità del differenziale.

Vediamo che $\mathcal{A}^\prime$ è un atlante orientato: le mappe di transizione si scrivono come
\begin{align*}
    (\phi_j)_*\circ(\phi_i)_*^{-1}:(V_i\cap V_j)\times\R^n&\to(V_i\cap V_j)\times\R^n\\
    (p,v)&\mapsto(\phi_j\circ\phi_i^{-1}(p),d(\phi_j\circ\phi_i^{-1})_p(v)),
\end{align*}
e posta $\psi := \phi_j\circ\phi_i^{-1}$, la matrice jacobiana di una tale mappa è triangolare inferiore a blocchi (come si vede dalla scrittura esplicita), con entrambi i blocchi
diagonali uguali a $\det(d\psi_p)$, e quindi il suo determinante è positivo in quanto quadrato dello Jacobiano di $\psi$.
\end{proof}

\textbf{Esercizio 4.5}
Dimostra l'identità di Jacobi: dati tre campi vettoriali $X,Y,Z$ su una varietà $M$, vale
\begin{equation*}
    [X,[Y,Z]] + [Y,[Z,X]] + [Z,[X,Y]] \equiv 0
\end{equation*}

\begin{proof}
\begin{align*}
    [X,[Y,Z]] + [Y,[Z,X]] + [Z,[X,Y]] &= (XY)Z - (YX)Z - Z(XY) + Z(YX) \\
    &+ (YZ)X - (ZY)X - X(YZ) + X(ZY) \\
    &+ (ZX)Y - (XZ)Y - Y(ZX) + Y(XZ) \\
    &= 0
\end{align*}
\end{proof}

\textbf{Esercizio 4.7}
Sia $M$ una varietà, siano $X,Y$ campi vettoriali su $M$ ed $f,g\in C^\infty(M)$. Mostra che
\begin{equation*}
    [fX,gY] = fg[X,Y] + f(Xg)Y - g(Yf)X
\end{equation*}

\begin{proof}
    Valutando il bracket in una funzione $h\in C^\infty(M)$ ed applicando la regola di Leibniz $X(fg) = (Xf)g + f(Xg)$, otteniamo
\begin{align*}
    [fX,gY]h &= fX(g\cdot Yh) - gY(f\cdot Xh) \\
    &= (f\cdot Xg)Yh + g(fXYh) - (g\cdot Yf)Xh - f(gYXh) \\
    &= fg[X,Y]h + f(Xg)Yh - g(Yf)Xh
\end{align*}
\end{proof}

\textbf{Esercizio 5.1}
Mostra che una distribuzione $D$ di rango 1 in una varietà $M$ è sempre integrabile.

\begin{proof}
    Per il Teorema di \textbf{Frobenius}, basta mostrare che la distribuzione $D$ è involutiva.
    Preso un punto $p\in M$ ed un suo intorno banalizzante $U$, visto che $D_p$ ha rango 1 un suo frame locale è dato da un qualche campo vettoriale $Z$ mai nullo su $U$ (sezione di $D$).

    Se $X,Y$ sono campi tangenti a $D$, localmente possiamo scriverceli come $X = fZ$ ed $Y = gZ$ per delle opportune funzioni lisce $f,g$. Allora, per l'esercizio 4.7 otteniamo

    \begin{align*}
        [X,Y] &= fg\cancel{[Z,Z]} + f(Zg)Z - g(Zf)Z \\
        &= (f(Zg) - g(Zf))Z,
    \end{align*}
    che valutato in ogni punto $q$ di $U$ (intorno banalizzante di $p$) finisce nello span di $Z(q)$, ovvero $D_q$; quindi $D$ è involutiva.
\end{proof}

\textbf{Esercizio 5.4}
Mostra che gli unici sottogruppi di Lie connessi di $SO(3)$ sono l'identità, $SO(3)$, e i sottogruppi isomorfi ad $S^1$ che descrivono le rotazioni intorno a un asse.

\begin{proof}
    Per usare il teorema di corrispondenza fra sottogruppi di Lie connessi e sottoalgebre di Lie, partiamo col classificare le sottoalgebre di Lie di $\mathfrak{so}(3) = \{X\in M_3(\R) \mid X^T = -X\}$.
    Visto che un base di
    $\mathfrak{so}(3)$ è data dalle matrici
    \begin{align*}
        X_1 = \begin{pmatrix}
            0 & 0 & 0 \\
            0 & 0 & -1 \\
            0 & 1 & 0
        \end{pmatrix}, \quad
        X_2 = \begin{pmatrix}
            0 & 0 & 1 \\
            0 & 0 & 0 \\
            -1 & 0 & 0
        \end{pmatrix}, \quad
        X_3 = \begin{pmatrix}
            0 & -1 & 0 \\
            1 & 0 & 0 \\
            0 & 0 & 0
        \end{pmatrix},
    \end{align*}
    mandando $X_i$ in $e_i$ si verifica che il bracket su $\mathfrak{so}(3)$ coincide con il prodotto vettoriale su $\R^3$, producendo quindi un isomorfismo di algebre di Lie. In
    tal modo risulta evidente che non ci siano sottoalgebre di dimensione 2, visto che il prodotto vettoriale di due vettori indipendenti dà un terzo vettore ortogonale a questi
    due, generando tutto $\R^3$.

    Perciò, escludendo l'identità ed $\mathfrak{so}(3)$ (che corrispondono rispettivamente ai sottogruppi di Lie $\{I\}$ ed $SO(3)$ stesso), ci siamo ridotti a mostrare che l'algebra di Lie
    di un sottogruppo $H$ di
    rotazioni intorno a un asse è del tipo $\mathfrak{h}=\{tM \mid t\in\R\}$ per una qualche $M\in\mathfrak{so}(3)$.

    Mostriamo preliminarmente che $\{\exp(tM) \mid t\in\R\}$ è un
    sottogruppo di Lie di $SO(3)$ isomorfo ad un tale $H$: infatti è l'immagine di $\mathfrak{h}$ tramite la restrizione della mappa esponenziale (visto che questa coincide con
    l'esponenziale di matrice sui sottogruppi di $GL(n)$), ma per corrispondenza esiste un sottogruppo di Lie con algebra di Lie $\mathfrak{h}$, e per la teoria della mappa
    esponenziale deduciamo che tale sottogruppo è proprio l'immagine della sua algebra tramite l'esponenziale, che è quindi un onesto sottogruppo di Lie.
    Più esplicitamente, l'esponenziale applicato della matrice
    $X_1$ di cui
    sopra produce una matrici di rotazione attorno ad $e_3$ (escono fuori gli sviluppi in serie di Taylor del seno e del coseno), e quindi $\{\exp(tM) \mid t\in\R\}$ è
    un sottogruppo di rotazioni intorno ad un asse perché $M$ è sempre simile ad un multiplo di $X_1$ (basta usare la forma normale di Jordan di $M$).

    Inoltre, per costruzione della mappa esponenziale è chiaro che l'algebra di Lie di $\{\exp(tM) \mid t\in\R\}$ sia $\mathfrak{h}$, e quindi abbiamo finito.
\end{proof}

\textbf{Esercizio 5.6}
Sia $\pi: E\to M$ un fibrato vettoriale. Mostra che $\pi$ è un'equivalenza omotopica.

\begin{proof}
L'inversa omotopica che cerchiamo è la sezione nulla $s_0:M\to E$:
\begin{align*}
    \begin{cases}
        \pi\circ s_0 = \id_M \\
        s_0\circ\pi \sim \id_E
    \end{cases}
\end{align*}

La prima relazione vale per definizione di $s_0$, in quanto manda ogni punto $p$ di $M$ in $(p,0)\in E_p$.
Per la seconda, costruiamo una omotopia $H:E\times I\to E$ fra $s_0\circ\pi$ ed $\id_E$ ponendo $H(e,t) = t\cdot e$, che è ben posta in quanto $E_p$ è uno
spazio vettoriale per ogni $p\in M$. Vediamo che è liscia: su un intorno banalizzante $U$ di $p$, se $E$ ha rango $k$ possiamo scrivere $H$ localmente come
\begin{align*}
    H:(U\times \R^k) \times I &\to U\times \R^k \\
    ((p,v),t) &\mapsto (p,t\cdot v),
\end{align*}

che è chiaramente liscia.
\end{proof}

\textbf{Esercizio 6.1}
Sia $M$ una varietà con bordo ed $N$ una varietà senza bordo. Mostra che $M\times N$ ha una naturale struttura di varietà con bordo, e che $\partial(M\times N)$ è diffeomorfo a $\partial M\times N$.

\begin{proof}
    Sicuramente il prodotto $M\times N$ è uno spazio topologico Hausdorff e secondo-numerabile. Se $\operatorname{dim}M = m$ e $\operatorname{dim}N = n$,  presi $\mathcal{A}= \{\phi_i:U_i\to V_i\}$ un $\R^m_+$-atlante per $M$ e $\mathcal{B}=\{\psi_j:W_j\to Z_j\}$ un atlante di $N$, l'insieme
    \[
        \{(\phi_i\times\psi_j):U_i\times W_j\to V_i\times Z_j\}
    \]
    è un atlante per $M\times N$: infatti $V_i\times Z_j$ è un aperto di $\R^m_+\times\R^n = \R^{m+n}_+$, e le mappe di transizione sono lisce per funtorialità del prodotto.

    Per quanto riguarda il bordo di $M\times N$, per definizione è dato dai punti che finiscono in $\partial \R^{m+n}_+$ tramite una qualche carta, quindi basta osservare che
    \[
        \partial (\R^m_+\times\R^n) = \partial \R^m_+\times\R^n,
    \]
    e che le carte di $\partial M \times N$ hanno valori in aperti del termine a destra.

% jump to next page
\pagebreak

\end{proof}

\textbf{Esercizio 6.4}
Mostra che una $n$-varietà $M$ è orientabile $\iff$ esiste una $n$-forma mai nulla su $M$.

\begin{proof} (svolto in collaborazione con Ludovico Piazza)

    Se $M$ è orientabile, per la Proposizione \textbf{7.2.15} delle dispense esiste una forma volume $\omega$ su $M$, che è mai nulla per definizione essendo strettamente positiva su ogni
    base positiva del tangente in un punto.

    Viceversa, se esiste una $n$-forma $\omega$ mai nulla su $M$, possiamo definire un'orientazione sui tangenti come segue: preso $p\in M$, esistono $v_1,\ldots,v_n\in T_pM$ tali
    che $\omega(p)(v_1,\ldots,v_n)$ è non nullo, ma in particolare questa deve essere una base del tangente, in quanto se ci fosse una relazione di dipendenza lineare, ad esempio $v_n = \sum_{i=1}^{n-1}\lambda_i v_i$, allora
    \[
        \omega(p)(v_1,\ldots,v_n) = \omega(p)(v_1,\ldots,v_{n-1},\sum_{i=1}^{n-1}\lambda_i v_i) = \sum_{i=1}^{n-1}\lambda_i\omega(p)(v_1,\ldots,v_i,\ldots,v_i) = 0
    \]
    per l'alternanza di $\omega$.

    A meno di cambiare segno al primo vettore della base, possiamo supporre che $\omega(p)(v_1,\ldots,v_n)$ sia strettamente positivo; poniamo quindi come orientazione positiva di
    $T_pM$ la classe di equivalenza di tale base.
    Osserviamo anche che, data un'altra base positiva $w_1,\ldots,w_n$ di $T_pM$, la matrice $A$ di cambio di base ha
    determinante positivo, e quindi $\omega(p)(w_1,\ldots,w_n) = \det(A)\omega(p)(v_1,\ldots,v_n) > 0$.

    Verifichiamo ora che tale scelta di orientazione è "localmente continua", cioè che per ogni $p\in M$ esiste una carta definita su un qualche intorno $U$ di $p$ il cui
    differenziale manda basi positive in basi positive.

    Preso un atlante $\mathcal{A}=\{\varphi_i:U_i\to \R^n\}$ per $M$, se fissiamo un punto $p\in M$ ed una
    carta $\varphi:
    U\to \R^n$ con $p\in U$, possiamo supporre che $d\varphi_p$ preservi l'orientazione di $T_pM$ (altrimenti componiamo la carta con una riflessione di $\R^n$). Leggendo $\omega$ in
    carte tramite $\varphi$, otteniamo una $n$-forma su $\R^n$
    \[
        \varphi_*\omega = f\cdot dx^1\wedge\ldots\wedge dx^n
    \]
    per qualche $f\in C^\infty(\R^n)$. Adesso, $f (\varphi (p)) > 0$ visto che per ogni base positiva $v_1,\ldots,v_n$ di $T_pM$ si ha
    \begin{align*}
        \omega (p)(v_1,\ldots,v_n) &= \varphi_*\omega(p)(d\varphi_p(v_1),\ldots,d\varphi_p(v_n))\\
        &= f(\varphi(p))\cdot (dx_1\wedge\ldots\wedge dx_n)(d\varphi_p(v_1),\ldots,d\varphi_p(v_n)) > 0
    \end{align*}
(essendo $dx_1\wedge\ldots\wedge dx_n$ strettamente positiva su basi positive di $\R^n$), e quindi possiamo supporre che $f\circ\varphi$ sia strettamente positiva su un qualche intorno $V\subset U$ di $p$.

    Questo implica che $d\varphi_q$ manda basi positive in basi positive per ogni $q\in V$: infatti per costruzione una base $w_1,\ldots,w_n$ di $T_qM$ è positiva se
    \begin{align*}
        \omega (q)(w_1,\ldots,w_n) &= \varphi_*\omega(q)(d\varphi_q(w_1),\ldots,d\varphi_q(w_n)) \\
        &= f(\varphi(q))\cdot (dx_1\wedge\ldots\wedge dx_n)(d\varphi_q(w_1),\ldots,d\varphi_q(w_n))
    \end{align*}
    è strettamente positivo, ma dato che $f(\varphi(q))>0$ deduciamo che $\{d\varphi_q(w_1),\ldots,d\varphi_q(w_n)\}$ è una base positiva di $\R^n$.
\end{proof}

\textbf{Esercizio 6.6}
Sia $M$ varietà qualsiasi ed $N$ varietà non orientabile. Il prodotto $M\times N$ può essere orientabile?

\begin{proof} (svolto in collaborazione con Ludovico Piazza)

    La risposta alla domanda è \textit{no}: mostriamo infatti che se $M\times N$ è orientabile allora $N$ è necessariamente orientabile.

    Per l'esercizio \textbf{6.4}, esiste una forma mai nulla $\omega$ su $M\times N$. Puntualmente, tale forma è non nulla su ogni base del tangente del prodotto, analogamente all'esercizio
    precedente. Fissato quindi $p\in M$ ed una base $\{v_1,\ldots,v_m\}$ di $T_pM$, possiamo definire una forma $\mu$ su $N$ (che identifichiamo con $\{p\}\times N  \subset M\times N$)
    come
    \[
        \mu(q)(w_1,\ldots,w_n) = \omega((p,q))(v_1,\ldots,v_m,w_1,\ldots,w_n),
    \]
    che è mai nulla per costruzione e quindi (sempre per l'esercizio precedente) otteniamo che $N$ è orientabile.
\end{proof}
\end{document}
